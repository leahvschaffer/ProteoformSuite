%!TEX root = ../proteoform_suite_manual.tex
%---------------------------------------------------------------------
%	DECONVOLUTION
%---------------------------------------------------------------------

\section{Deconvolution}

Results from deconvolution of MS1 spectra are used to identify proteoforms by intact-mass analysis and to construct proteoform families of observed proteoforms. Deconvolution results are loaded on the Load Results page (see \textbf{Load Results: Standard}) under Deconvolution Results for Identification and Deconvolution Results for Quantification. This section describes how to obtain deconvolution results to import into Proteoform Suite.

\subsection{Thermo Deconvolution 4.0}
\begin{itemize}
\item See Thermo Fisher website for quote and user guide
\item Run Xtract algorithm for high resolution data
\item For each .raw file:
\begin{itemize}
	\item Select Open Results in the Run Queue
	\item Right-click the Results table and choose Export All
	\item Open the .xls file exported and save as a .xlsx file
\end{itemize}
\item The saved .xlsx file is used for Deconvolution Results for Identification and Deconvolution Results for Quantification in the Standard analysis on the Load Results page
\end{itemize}

\subsection{FLASHDeconv}
FLASHDeconv is an ultra-fast deconvolution algorithm for high resolution mass spectrometry data developed by the OpenMS team.\supercite{Jeong2020}
\begin{itemize}
\item On the Load Results page, select FLASHDeconv Deconvolution under Choose Analysis (top left)
\item Input .mzML files into the table (Load Data drop down menu will be set to Spectra Files)
\item FLASHDeconv requires .mzML files. Use MSConvert to convert other file types (\url{http://proteowizard.sourceforge.net/}). Peak picking is NOT recommended.
\item Set Parameters:
\begin{itemize}
\item Min Charge: minimum charge state allowed for deconvolution
\item Max Charge: maximum charge state allowed for deconvolution
\end{itemize}
\item To begin deconvolution, hit the Deconvolute button under Start Analysis (bottom right)
\item For more advanced parameter options, you can also run the command line version of FLASHDeconv, available at \url{https://www.openms.de/comp/flashdeconv/}
\item The resulting .tsv file is used for Deconvolution Results for Identification and Deconvolution Results for Quantification in the Standard analysis on the Load Results page
\end{itemize}

\subsection{Other}
\begin{itemize}
\item If you have another deconvolution algorithm of choice, simply create a three column tab-separated file
\subitem Monoisotopic mass \textbackslash t intensity \textbackslash t retention time
\item This .tsv or .txt file can be used for Deconvolution Results for Identification and Deconvolution Results for Quantification in the Standard analysis on the Load Results page
\end{itemize}